\begin{center}
  \large\textbf{ABSTRACT}
\end{center}

\addcontentsline{toc}{chapter}{ABSTRACT}

\vspace{2ex}

\begingroup
  % Menghilangkan padding
  \setlength{\tabcolsep}{0pt}

  \noindent
  \begin{tabularx}{\textwidth}{l >{\centering}m{3em} X}
    % Ubah kalimat berikut dengan nama mahasiswa
    \emph{Name}     &:& Agung Wicaksono \\

    % Ubah kalimat berikut dengan judul tugas akhir dalam Bahasa Inggris
    \emph{Title}    &:& \emph{Pedestrian Detection on \textit{Zebracross} for Car Driver Early Warning using \textit{Mask R-CNN}} \\

    % Ubah kalimat-kalimat berikut dengan nama-nama dosen pembimbing
    \emph{Advisors}  &:& 1. Prof. Dr. Ir. Mauridhi Hery Purnomo, M.Eng. \\
    				 & & 2. Dr. Eko Mulyanto Yuniarno, S.T., M.T. \\
  \end{tabularx}
\endgroup

% Ubah paragraf berikut dengan abstrak dari tugas akhir dalam Bahasa Inggris
\emph{Today, safety features on four-wheeled vehicles or cars have developed very rapidly. This is evidenced by the number of car manufacturers that apply seat belt technology, air bags, adaptive cruise control, electronic stability control, autonomous emergency braking, blind spot monitoring and so on. However, the features mentioned above are still considered less friendly for pedestrians. It is proven that according to data from the WHO, there are 270,000 pedestrians who die every year or about 22\% of all victims die due to road accidents. Starting from these problems, the author will conduct research on the detection of pedestrians at zebracross for early warning car drivers as a research topic. In this final project, there are 3 objects to be detected, namely pedestrians, zebracross and motorcyclists using the Mask R-CNN method. The best results obtained are the use of \textit{ResNet-101} for \textit{backbone Mask R-CNN} with a score of \textit{mAP} of 76,605\%, mAR of 85.375\% and \textit{F1-Score} of 80.302\% .}

% Ubah kata-kata berikut dengan kata kunci dari tugas akhir dalam Bahasa Inggris
\emph{Keywords}: \emph{Pedestrian, Zebracross, Mask R-CNN, Image Processing}
