\chapter{TINJAUAN PUSTAKA}
\label{chap:tinjauanpustaka}

% Ubah bagian-bagian berikut dengan isi dari tinjauan pustaka

Demi mendukung penelitian ini, dibutuhkan beberapa teori penunjang sebagai bahan acuan dan refrensi. Dengan demikian penelitian ini menjadi lebih terarah.

\section{Dasar Teori}
\label{sec:dasarteori}

\subsection{\textit{Artificial Intelligence}}
\label{artificialintelligence}

\textit{Artificial Intelligence} mengacu pada simulasi kecerdasan manusia dalam mesin yang diprogram untuk berpikir seperti manusia dan meniru tindakan mereka.\citep{artificialintellingece} Istilah ini juga dapat diterapkan pada mesin apa pun yang menunjukkan ciri-ciri yang terkait dengan pikiran manusia seperti pembelajaran dan pemecahan masalah.

Karakteristik ideal dari \textit{artificial intellingence} adalah kemampuannya untuk merasionalisasi dan mengambil tindakan yang memiliki peluang terbaik untuk mencapai tujuan tertentu. Bagian dari \textit{artificial intellingece} adalah \textit{machine learning}, yang mengacu pada konsep bahwa program komputer dapat secara otomatis belajar dari dan beradaptasi dengan data baru tanpa dibantu oleh manusia. Teknik\textit{deep learning} memungkinkan pembelajaran otomatis ini melalui penyerapan sejumlah besar data tidak terstruktur seperti teks, gambar, atau video.

\subsection{\textit{Machine Learning}}
\label{machinelearning}

\textit{Machine Learning} adalah studi tentang algoritma komputer yang memberikan sistem kemampuan untuk belajar secara otomatis dan dapat meningkatkan kemampuan dari pengalaman yang sudah didapatkan \citep{machinelearning1}. Hal ini umumnya dilihat sebagai sub-bidang kecerdasan buatan. Algoritma pembelajaran mesin memungkinkan sistem membuat keputusan secara mandiri tanpa dukungan eksternal. Keputusan semacam itu dibuat dengan menemukan pola dasar yang berharga dalam data yang kompleks. Berdasarkan pendekatan pembelajaran, jenis data \textit{input} dan \textit{output}, dan jenis masalah yang dipecahkan, ada beberapa kategori utama dari algoritma \textit{machine learning} \textit{supervised, unsupervised} dan \textit{reinforcement learning}. Ada beberapa pendekatan hibrida dan metode umum lainnya yang menawarkan ekstrapolasi alami dari bentuk masalah pembelajaran mesin. Berikut merupakan penjelasan dari beberapa kategori utama dari algoritma \textit{machine learning}:

\begin{enumerate}
	\item \textit{Supervised Learning} diterapkan ketika data dalam bentuk variabel input dan nilai target output. Algoritma akan mempelajari fungsi pemetaan dari \textit{input} ke \textit{output}. Ketersediaan sampel data berlabel dengan skala besar mempunyai nilai yang tinggi dikarenakan masih terdapat kelangkaan \textit{dataset}. Pendekatan ini secara luas dapat dibagi menjadi dua kategori utama yaitu \textit{classification} dan \textit{regression}. Gambar \ref{fig:supervised} menampilkan visualisasi dari \textit{classification} dan \textit{regression} pada \textit{Supervised Learning}
	
	\begin{figure}[ht]
		\centering
		\includegraphics[scale=0.1]{gambar/supervised.png}
		\caption{Gambaran \textit{Supervised Learning}\citep{supervised}}
		\label{fig:supervised}
	\end{figure}  
	
	\item \textit{Unsupervised Learning} diterapkan ketika data hanya tersedia dalam bentuk \textit{input} dan tidak ada variabel \textit{output} yang sesuai. Algoritma semacam itu memodelkan pola yang mendasari data untuk mempelajari lebih lanjut tentang karakteristiknya. Salah satu jenis utama dari algoritma \textit{unsupervised} adalah pengelompokan. Dalam teknik ini, kelompok yang melekat dalam data ditemukan dan kemudian digunakan untuk memprediksi \textit{output} untuk \textit{input} yang tidak terlihat. Contoh dari teknik ini adalah untuk memprediksi perilaku pembelian pada pelanggan. Gambar \ref{fig:unsupervised} merupakan visualisasi dari algoritma \textit{unsupervised learning}.
	
	\begin{figure}[ht]
		\centering
		\includegraphics[scale=0.2]{gambar/unsupervised.jpg}
		\caption{Gambaran \textit{Unsupervised Learning}\citep{unsupervised}}
		\label{fig:unsupervised}
	\end{figure}
	
	
	\item \textit{Reinforcement learning} diterapkan ketika tugas yang ada
	adalah membuat urutan keputusan menuju \textit{reward} akhir. Selama proses \textit{learning}, \textit{artificial agent} mendapat \textit{reward} atau \textit{penalties} atas tindakan yang dilakukannya. Tujuannya adalah untuk memaksimalkan total \textit{reward} yang didapatkan. Gambar \ref{fig:reinforcement} merupakan visualisasi dari algoritma \textit{reinforcement learning}.
	
	\begin{figure}[ht]
		\centering
		\includegraphics[scale=0.3]{gambar/reinforcement.png}
		\caption{Gambaran \textit{Reinforcement Learning}\citep{reinforcement}}
		\label{fig:reinforcement}
	\end{figure}
\end{enumerate}

\subsection{\textit{Deep Learning}}
\label{deeplearning}

\textit{Deep Learning} adalah kelas \textit{machine learning} yang berkinerja jauh lebih baik pada data tidak terstruktur\citep{dl}. Teknik \textit{deep learning} mengungguli teknik \textit{machine learning} saat ini. Ini memungkinkan model komputasi untuk mempelajari fitur secara progresif dari data di berbagai level. Popularitas \textit{deep learning} diperkuat karena jumlah data yang tersedia meningkat serta kemajuan perangkat keras yang menyediakan komputer yang kuat.

Arsitektur \textit{deep learning} berkinerja lebih baik daripada jaringan saraf tiruan  sederhana, meskipun waktu \textit{learning} dari struktur \textit{deep learning} lebih tinggi dari jaringan saraf tiruan. Namun, waktu \textit{learning} dapat dikurangi dengan menggunakan metode seperti \textit{transfer learning} atau komputasi menggunakan GPU. Salah satu faktor yang menentukan keberhasilan jaringan saraf terletak pada desain arsitektur jaringan yang cermat.

\subsection{\textit{Convolutional Neural Network}}
\label{cnn}

\textit{Convolutional Neural Networl} (CNN) adalah jenis khusus dari \textit{multilayer neural network} atau arsitektur \textit{deep learning} yang terinspirasi oleh sistem visual makhluk hidup \citep{cnn}. CNN sangat cocok untuk berbagai bidang visi komputer dan \textit{natural language processing}. \textit{Convolutional Neural Network} (CNN), juga disebut \textit{ConvNet}, adalah jenis \textit{Artificial Neural Network} (ANN), yang memiliki arsitektur \textit{feed-forward} yang dalam dan memiliki kemampuan generalisasi yang luar biasa dibandingkan dengan jaringan lain dengan lapisan FC (\textit{Fully Connected}), ia dapat mempelajari fitur objek yang sangat abstrak terutama data spasial dan dapat mengidentifikasinya dengan lebih efisien. Model CNN yang dalam terdiri dari satu set lapisan pemrosesan yang dapat mempelajari berbagai fitur data \textit{input} (misalnya gambar) dengan beberapa tingkat abstraksi seperti yang ditampilkan pada Gambar \ref{fig:concept-cnn}. Lapisan inisiator mempelajari dan mengekstrak fitur tingkat tinggi (dengan abstraksi yang lebih rendah), dan lapisan yang lebih dalam mempelajari dan mengekstrak fitur tingkat rendah (dengan abstraksi yang lebih tinggi).

\begin{figure}[ht]
	\centering
	\includegraphics[scale=0.3]{gambar/concept-cnn.png}
	\caption{Gambaran Konsep Arsitektur CNN \citep{concept-cnn}}
	\label{fig:concept-cnn}
\end{figure}

\textit{Convolutional Neural Network} memiliki beberapa keunggulan dibanding dengan jaringan saraf tiruan lainnya dalam konteks visi komputer, antara lain :

\begin{enumerate}
	\item Salah satu alasan utama untuk mempertimbangkan CNN dalam kasus tersebut adalah fitur pembagian bobot dari CNN, yang mengurangi jumlah parameter yang dapat dilatih dalam jaringan, yang membantu model untuk menghindari \textit{overfitting} dan juga untuk meningkatkan generalisasi.
	\item Pada CNN, lapisan klasifikasi dan lapisan ekstraksi fitur melakukan proses \textit{learning} secara bersama-sama, yang membuat output model lebih terorganisir dan membuat output lebih bergantung pada fitur yang diekstraksi.
	\item Implementasi pada jaringan dengan ukuran yang besar akan lebih sulit dilakukan dengan menggunakan jenis jaringan saraf lain daripada menggunakan \textit{Convolutional Neural Network}
\end{enumerate}

Convolutional Neural Network memiliki beberapa layer antara lain:

\begin{enumerate}
	\item \textit{Convolutional layer} adalah komponen terpenting dari arsitektur CNN mana pun. Ini berisi satu set kernel convolutional (juga disebut filter), yang dililitkan dengan gambar input (metrik N-dimensi) untuk menghasilkan peta fitur keluaran. Kernel dapat digambarkan sebagai kisi nilai atau angka diskrit, di mana setiap nilai dikenal sebagai bobot kernel. Selama awal proses pelatihan model CNN, semua bobot kernel ditetapkan dengan angka acak (pendekatan yang berbeda juga tersedia untuk inisialisasi bobot). Kemudian, dengan setiap periode \textit{learning}, bobot disetel dan kernel belajar mengekstrak fitur yang memberikan informasi mengenai data.
	
	\item \textit{Pooling layer} digunakan untuk membuat sub-sampel peta fitur (dihasilkan setelah operasi konvolusi), yaitu mengambil peta fitur berukuran lebih besar dan mengecilkannya menjadi peta fitur berukuran lebih rendah. Saat menyusutkan peta fitur, ia selalu mempertahankan fitur (atau informasi) yang paling dominan di setiap langkah \textit{pools}. Operasi \textit{pooling} dilakukan dengan menentukan ukuran \textit{pooled region} dan langkah operasi, mirip dengan operasi konvolusi. Ada berbagai jenis teknik \textit{pooling} yang digunakan dalam berbagai \textit{pooling layer} seperti \textit{max pooling, min pooling, average pooling, gated pooling, tree pooling}, dan lain-lain. \textit{Max Pooling} adalah teknik pooling yang paling populer dan banyak digunakan.
\end{enumerate}


\section{Penelitian Terkait}
\label{penelitianterkait}

\subsection{Real-Time Pedestrian Detection With Deep Network Cascades}
\label{realtime pedestrian}

Penelitian ini dilakukan oleh Anelia Angelova dan kawan-kawan pada tahun 2015 yang menyajikan pendekatan real-time baru untuk deteksi objek yang mengeksploitasi efisiensi \textit{cascade classifiers} dengan akurasi \textit{deep neural network} \citep{penelitianterkait1}. \textit{Deep network} telah terbukti unggul dalam tugas klasifikasi, dan kemampuannya untuk beroperasi pada \textit{raw pixel input} tanpa perlu merancang fitur khusus. Namun, \textit{deep network} terkenal lambat pada waktu inferensi. Dalam \textit{paper} tersebut, Anelia Angelova dan kawan-kawan mengusulkan pendekatan \textit{cascades deep network} dan \textit{fast features}, yang sangat cepat dan sangat akurat. Mereka menerapkannya pada permasalahan pada deteksi pejalan kaki. Algoritma mereka berjalan secara real-time pada 15 frame per detik. Pendekatan yang dihasilkan mencapai tingkat kesalahan rata-rata 26,2\% pada \textit{benchmark} deteksi Caltech Pedestrian.