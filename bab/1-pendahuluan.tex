\chapter{PENDAHULUAN}
\label{chap:pendahuluan}

% Ubah bagian-bagian berikut dengan isi dari pendahuluan

Penelitian ini di latar belakangi oleh berbagai kondisi yang menjadi acuan. Selain itu juga terdapat beberapa permasalahan yang akan dijawab sebagai luaran dari penelitian.

\section{Latar Belakang}
\label{sec:latarbelakang}

Mobil merupakan salah satu jenis kendaraan bermotor yang banyak terdapat di Indonesia. Pada tahun 2018 Badan Pusat Statistik mencatat terdapat 16.440.987 mobil penumpang yang berada di Indonesia. Dengan bertambahnya jumlah mobil di Indonesia dari tahun ke tahun, meningkatkan juga jumlah kecelakaan mobil. Fitur keselamatan dan keamanan pada mobil sangat penting bagi para pengendara dan penumpang, sehingga para produsen mobil berusaha meningkatkan teknologi keselamatan dan keamanan pada mobil buatannya. Sebagai contoh beberapa fitur keselamatan dan keamanan yang
terdapat pada mobil antara lain, adaptive cruise control, hill strat assist, blind spot monitoring, electronic stability control dan lain sebagainya.

Menurut data dari WHO, terdapat 270.000 pejalan kaki meninggal dunia setiap tahun atau sekitar 22\% dari seluruh korban meniggal akibat kecelakan di jalan. Melihat kegiatan para pejalan kaki yang jarang berada di badan jalan, angka tersebut tentu cukup tinggi. Para pejalan kaki hanya menggunakan badan jalan ketika hendak menyebrang jalan lewat zebracross. Kelalaian dari pejalan kaki maupun pengendara mobil merupakan faktor utama mengapa angka kematian pejalan kaki cukup tinggi. Salah satu contoh kelalaian pejalan kaki adalah pada saat menyebrang jalan tidak memperhatikan kendaraan yang akan lewat dan atau melihat rambu serta lampu lalu lintas. Di sisi pengendara mobil, kelelahan,
kurangnya fokus saat berkendara dan tidak memperhatikan rambu maupun marka dapat berakibat fatal
baik kepada pejalan kaki dan pengendara lain.

Teknologi artificial intelligent sudah banyak disematkan pada mobil pada masa kini, dibuktikan
dengan adanya teknologi adaptive cruise control, hill start assist dan lain sebagainya. Artificial intelligent khususnya deep learning tentu dapat digunakan untuk deteksi pejalan kaki di zebracross guna mengurangi jumlah korban akibat kecelakaan. Deteksi pejalan kaki dapat digabungkan dengan buzzer dan atau LED sebagai komponen output untuk mengingatkan kepada pengendara bahwa ada pejalan kaki yang sedang menyebrangi jalan serta mengembalikan fokus untuk berkendara.

\section{Permasalahan}
\label{sec:permasalahan}

Cukup tingginya angka kematian pejalan kaki akibat kecelakaan lalu lintas dan belum adanya deteksi pejalan kaki di zebracross untuk peringatan dini kepada pengendara mobil. Oleh karena itu, diperlukan sebuah sistem yang mampu mendeteksi adanya pejalan kaki yang berada disekitar jalan raya untuk selanjutnya dapat digunakan sebagai peringatan kepada pengendara mobil.

\section{Tujuan}
\label{sec:Tujuan}

Berdasarkan rumusan permasalahan di atas, tujuan dari penelitian ini adalah untuk mendeteksi dan mensegmentasikan pejalan kaki dan \textit{zebracross} di jalan raya untuk peringatan dini pengendara mobil menggunakan metode \textit{Mask R-CNN}.



\section{Batasan Masalah}
\label{sec:batasanmasalah}

Batasan masalah yang timbul dari permasalahan Tugas Akhir ini adalah:

\begin{enumerate}[nolistsep]

  \item Menggunakan Mask R-CNN untuk pendeteksian pejalan kaki dan zebracross
  \item File Input berupa video dengan format MP4 dengan 30 fps. 

\end{enumerate}

\section{Sistematika Penulisan}
\label{sec:sistematikapenulisan}

Laporan penelitian tugas akhir ini tersusun dalam sistematika dan terstruktur sehingga mudah dipahami dan dipelajari oleh pembaca maupun seseorang yang ingin melanjutkan penelitian ini. Alur sistematika penulisan laporan penelitian ini yaitu :

\begin{enumerate}[nolistsep]

  \item \textbf{BAB I Pendahuluan}

  Bab ini berisi uraian tentang latar belakang permasalahan, penegasan dan alasan pemilihan judul, sistematika laporan, tujuan dan metodologi penelitian.

  \vspace{2ex}

  \item \textbf{BAB II Tinjauan Pustaka}

  Pada bab ini berisi tentang uraian secara sistematis teori-teori yang berhubungan dengan permasalahan yang dibahas pada penelitian ini. Teori-teori ini digunakan sebagai dasar dalam penelitian, yaitu informasi terkait pejalan kaki, \textit{zebracross}, algoritma \textit{Mask RCNN}, dan teori-teori penunjang lainya.

  \vspace{2ex}

  \item \textbf{BAB III Desain dan Implementasi Sistem}

  Bab ini berisi tentang penjelasan-penjelasan terkait eksperimen yang akan dilakukan dan langkah-langkah pengambilan data jalan raya serta proses deteksi pejalan kaki pada \textit{zebracross}. Guna mendukung hal tersebut, digunakanlah blok diagram atau work flow agar sistem yang akan dibuat dapat terlihat dan mudah dibaca untuk implentasi pada pelaksanaan tugas akhir.

  \vspace{2ex}

  \item \textbf{BAB IV Pengujian dan Analisa}

  Bab ini menjelaskan tentang pengujian eksperimen yang dilakukan terhadap citra jalan raya, proses klasifikasi pejalan kaki dan \textit{zebracross}. Serta terkait tingkat akurasi keberhasilan pengujian yang dilengkapi dengan analisanya.

  \vspace{2ex}

  \item \textbf{BAB V Penutup}

  Bab ini merupakan penutup yang berisi kesimpulan yang di ambil dari penelitian dan pengujian yang telah dilakukan. Saran dan kritik yang membangun untuk pengembangkan lebih lanjut juga dituliskan pada bab ini.

\end{enumerate}
