\chapter{PENUTUP}
\label{chap:penutup}

% Ubah bagian-bagian berikut dengan isi dari penutup

\section{Kesimpulan}
\label{sec:kesimpulan}

Berdasarkan hasil pengujian yang telah dilakukan, penulis dapat menyimpulkan beberapa hal sebagai berikut:

\begin{enumerate}[nolistsep]

  \item Dalam penelitian ini telah diimplementasikan dengan baik proses deteksi dan segmentasi pejalan kaki dan zebra cross dengan menggunakan Mask R-CNN, dengan \textit{mean Average Precision} sebesar 90,476\%.
  \item \textit{Backbone} ResNet-101 memiliki hasil akurasi yang lebih baik dibanding dengan \textit{backbone} lainnya dengan peforma lebih tinggi sebesar 16,92\% dibanding ResNet-50 dan 71,73\% dibanding MobileNet-v1. 
  \item Peforma pendeteksian dapat berjalan baik dalam beberapa kondisi waktu apabila memiliki cukup pencahayaan..
  
\end{enumerate}

\section{Saran}
\label{sec:saran}

Untuk pengembangan lebih lanjut pada penelitian mendatang, maka penulis memiliki saran sebagai berikut:

\begin{enumerate}[nolistsep]

  \item Menambah jumlah \textit{dataset} yang masih sedikit pada kelas pejalan kaki dan \textit{zebracross}.

  \item Pembuatan dataset pejalan kaki di Indonesia karena perilaku pejalan kaki antar negara memiliki perbedaan yang cukup signifikan.

\end{enumerate}
