\chapter{PENUTUP}
\label{chap:penutup}

% Ubah bagian-bagian berikut dengan isi dari penutup

\section{Kesimpulan}
\label{sec:kesimpulan}

Berdasarkan hasil pengujian yang telah dilakukan, penulis dapat menyimpulkan beberapa hal sebagai berikut:

\begin{enumerate}[nolistsep]

  \item Dalam penelitian ini telah diimplementasikan dengan baik proses pendeteksian pejalan kaki dan zebracross dengan menggunakan Mask R-CNN, dengan \textit{mean Average Precision} sebesar 76.62\%.
  \item \textit{Backbone} ResNet-101 memiliki hasil akurasi yang lebih baik dibanding dengan \textit{backbone} lainnya dengan peforma lebih tinggi sebesar 12.82\% dibanding ResNet-50 dan 67.36\% dibanding MobileNet-v1. 
  \item Waktu yang dibutuhkan dalam proses pendektesian akan semakin lama jika objek yang berada pada gambar semakin banyak.
  
\end{enumerate}

\section{Saran}
\label{sec:saran}

Untuk pengembangan lebih lanjut pada penelitian mendatang, maka penulis memiliki saran sebagai berikut:

\begin{enumerate}[nolistsep]

  \item Menambah jumlah sample data pada kelas pengendara motor yang dinilai masih sedikit untuk dataset yang dipilih.

  \item Pembuatan dataset pejalan kaki di Indonesia karena perilaku pejalan kaki antar negara memiliki perbedaan yang cukup signifikan.

\end{enumerate}
