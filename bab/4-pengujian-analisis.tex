\chapter{PENGUJIAN DAN ANALISIS}
\label{chap:pengujiananalisis}

% Ubah bagian-bagian berikut dengan isi dari pengujian dan analisis

Pada bab ini dipaparkan hasil pengujian serta analisa dari desain sistem dan implementasi. Pengujian dilakukan guna mengetahui tingkat kesalahan dan menarik kesimpulan dari sistem yang telah dibuat.

Pada proses oengujian digunakan salah satu layanan \textit{Google} yaitu \textit{Google Colaboratory} dengan spesifikasi \textit{hardware} seperti pada Tabel \ref{tab:spek-colab}. Sedangkan untuk spesifikasi \textit{hardware} komputer penulis dapat dilihat pada Tabel \ref{tab:spek-pc}.
\begin{table}[h!]
	\begin{center}
		\begin{tabular}{ |c|c| } 
			\hline
			\textbf{Procesor} & Intel Xeon Processor @ 2.3 GHz\\
			\hline 
			\textbf{Graphic Card} & Tesla K80 12 GB GDDR5 VRAM\\
			\hline 
			\textbf{RAM} & 16 GB\\ 
			\hline
		\end{tabular}
		\caption{Spesifikasi \textit{hardware Google Colaboratory}}
		\label{tab:spek-colab}
	\end{center}
\end{table} 

\begin{table}[h!]
	\begin{center}
		\begin{tabular}{ |c|c| } 
			\hline
			\textbf{Procesor} & Intel(R) Core(TM) i5-10400F CPU @ 2.90GHz\\
			\hline 
			\textbf{Graphic Card} & Nvidia GeForce GTX 1650 4 GB GDDR6\\
			\hline 
			\textbf{RAM} & 8 GB\\ 
			\hline
		\end{tabular}
		\caption{Spesifikasi \textit{hardware} Komputer yang Digunakan}
		\label{tab:spek-pc}
	\end{center}
\end{table}

\section{Skenario Pengujian}
\label{sec:skenariopengujian}

Pengujian dilakukan dengan \lipsum[1-2]

\section{Evaluasi Pengujian}
\label{sec:analisispengujian}

Dari pengujian yang \lipsum[1]

% Contoh pembuatan tabel
\begin{longtable}{|c|c|c|}
  \caption{Hasil Pengukuran Energi dan Kecepatan}
  \label{tb:EnergiKecepatan}\\
  \hline
  \rowcolor[HTML]{C0C0C0}
  \textbf{Energi} & \textbf{Jarak Tempuh} & \textbf{Kecepatan} \\
  \hline
  10 J & 1000 M & 200 M/s \\
  20 J & 2000 M & 400 M/s \\
  30 J & 4000 M & 800 M/s \\
  40 J & 8000 M & 1600 M/s \\
  \hline
\end{longtable}

\lipsum[2-4]
