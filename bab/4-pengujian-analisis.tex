\chapter{PENGUJIAN DAN ANALISIS}
\label{chap:pengujiananalisis}

% Ubah bagian-bagian berikut dengan isi dari pengujian dan analisis

Pada bab ini dipaparkan hasil pengujian serta analisa dari desain sistem dan implementasi. Pengujian dilakukan guna mengetahui tingkat kesalahan dan menarik kesimpulan dari sistem yang telah dibuat.

Pada proses pengujian digunakan salah satu layanan \textit{Google} yaitu \textit{Google Colaboratory} dengan spesifikasi \textit{hardware} seperti pada Tabel \ref{tab:spek-colab}. Sedangkan untuk spesifikasi \textit{hardware} komputer penulis dapat dilihat pada Tabel \ref{tab:spek-pc}.
\begin{table}[h!]
	\begin{center}
		\begin{tabular}{ |c|c| } 
			\hline
			\textbf{Procesor} & Intel Xeon Processor @ 2.3 GHz\\
			\hline 
			\textbf{Graphic Card} & Tesla K80 12 GB GDDR5 VRAM\\
			\hline 
			\textbf{RAM} & 16 GB\\ 
			\hline
		\end{tabular}
		\caption{Spesifikasi \textit{hardware Google Colaboratory}}
		\label{tab:spek-colab}
	\end{center}
\end{table} 

\begin{table}[h!]
	\begin{center}
		\begin{tabular}{ |c|c| } 
			\hline
			\textbf{Procesor} & Intel(R) Core(TM) i5-10400F CPU @ 2.90GHz\\
			\hline 
			\textbf{Graphic Card} & Nvidia GeForce GTX 1650 4 GB GDDR6\\
			\hline 
			\textbf{RAM} & 8 GB\\ 
			\hline
		\end{tabular}
		\caption{Spesifikasi \textit{hardware} Komputer yang Digunakan}
		\label{tab:spek-pc}
	\end{center}
\end{table}

Pengujian dilakukan dengan membagi model ke beberapa jenis \textit{backbone} yang digunakan, antara lain Resnet-50, Resnet-101, dan Mobilenet-V1.

\section{Pengujian Jenis \textit{Backbone}}
\label{sec:pengujian-backbone}

Pengujian pada jenis \textit{backbone} bertujuan untuk mengetahui performa dan akurasi dari setiap model yang dihasilkan dengan \textit{backbone} yang berbeda.

\subsection{Resnet 50}
\label{subsec:resnet50}

Tabel \ref{tab:conf-resnet50} merupakan parameter-parameter yang digunakan untuk membuat model Mask R-CNN dengan menggunakan \textit{backbone} Resnet-50.

\begin{longtable}[h!]{|l|l|}
		\hline
		\multicolumn{2}{|c|}{\textbf{Pengaturan Model Resnet-50}}                                                                                                                                                                         \\ \hline
		BACKBONE                        & resnet50                                                                                                                                                                              \\ \hline
		BACKBONE\_STRIDES               & {[}4, 8, 16, 32, 64{]}                                                                                                                                                                 \\ \hline
		BATCH\_SIZE                     & 1                                                                                                                                                                                      \\ \hline
		BBOX\_STD\_DEV                  & {[}0.1 0.1 0.2 0.2{]}                                                                                                                                                                  \\ \hline
		COMPUTE\_BACKBONE\_SHAPE        & None                                                                                                                                                                                   \\ \hline
		DETECTION\_MAX\_INSTANCES       &	 50                                                    	\\ \hline
		IMAGE\_META\_SIZE               & 16                                                                                                                                                                                     \\ \hline
		IMAGE\_MIN\_DIM                 & 400                                                                                                                                                                                    \\ \hline
		IMAGE\_MIN\_SCALE               & 0                                                                                                                                                                                      \\ \hline
		IMAGE\_RESIZE\_MODE             & square                                                                                                                                                                                 \\ \hline
		IMAGE\_SHAPE                    & {[}512 512{]}                                                                                                                                                                             \\ \hline
		LEARNING\_MOMENTUM              & 0.9                                                                                                                                                                                    \\ \hline
		LEARNING\_RATE                  & 0.001                                                                                                                                                                                  \\ \hline
		LOSS\_WEIGHTS                   & \begin{tabular}[c]{@{}l@{}}\{'rpn\_class\_loss': 1.0, \\ 'rpn\_bbox\_loss': 1.0, \\ 'mrcnn\_class\_loss': 1.0, \\ 'mrcnn\_bbox\_loss': 1.0, \\ 'mrcnn\_mask\_loss': 1.0\}\end{tabular} \\ \hline
		MASK\_POOL\_SIZE                & 14                                                                                                                                                                                     \\ \hline
		MASK\_SHAPE                     & {[}28, 28{]}                                                                                                                                                                           \\ \hline
		MAX\_GT\_INSTANCES              & 50                                                                                                                                                                                     \\ \hline
		MEAN\_PIXEL                     & {[}123.7 116.8 103.9{]}                                                                                                                                                                \\ \hline
		MINI\_MASK\_SHAPE               & (56, 56)                                                                                                                                                                               \\ \hline
		NAME                            & object                                                                                                                                                                                 \\ \hline
		NUM\_CLASSES                    & 4                                                                                                                                                                                      \\ \hline
		POOL\_SIZE                      & 7                                                                                                                                                                                      \\ \hline
		MASK\_SHAPE                     & {[}28, 28{]}                                                                                                                                                                           \\ \hline
		MAX\_GT\_INSTANCES              & 50                                                                                                                                                                                     \\ \hline
		MEAN\_PIXEL                     & {[}123.7 116.8 103.9{]}                                                                                                                                                                \\ \hline
		MINI\_MASK\_SHAPE               & (56, 56)                                                                                                                                                                               \\ \hline
		NAME                            & object                                                                                                                                                                                 \\ \hline
		NUM\_CLASSES                    & 4                                                                                                                                                                                      \\ \hline
		POOL\_SIZE                      & 7                                                                                                                                                                                      \\ \hline
		POST\_NMS\_ROIS\_INFERENCE      & 1000                                                                                                                                                                                   \\ \hline
		POST\_NMS\_ROIS\_TRAINING       & 2000                                                                                                                                                                                   \\ \hline
		PRE\_NMS\_LIMIT                 & 6000                                                                                                                                                                                   \\ \hline
		ROI\_POSITIVE\_RATIO            & 0.33                                                                                                                                                                                   \\ \hline
		RPN\_ANCHOR\_RATIOS             & {[}0.5, 1, 2{]}                                                                                                                                                                        \\ \hline
		RPN\_ANCHOR\_SCALES             & (32, 64, 128, 256, 512)                                                                                                                                                                \\ \hline
		RPN\_ANCHOR\_STRIDE             & 1                                                                                                                                                                                      \\ \hline
		RPN\_BBOX\_STD\_DEV             & {[}0.1 0.1 0.2 0.2{]}                                                                                                                                                                  \\ \hline
		RPN\_NMS\_THRESHOLD             & 0.7                                                                                                                                                                                    \\ \hline
		RPN\_TRAIN\_ANCHORS\_PER\_IMAGE & 256                                                                                                                                                                                    \\ \hline
		STEPS\_PER\_EPOCH               & 100                                                                                                                                                                                    \\ \hline
		TOP\_DOWN\_PYRAMID\_SIZE        & 256                                                                                                                                                                                    \\ \hline
		TRAIN\_BN                       & False                                                                                                                                                                                  \\ \hline
		TRAIN\_ROIS\_PER\_IMAGE         & 200                                                                                                                                                                                    \\ \hline
		USE\_MINI\_MASK                 & True                                                                                                                                                                                   \\ \hline
		USE\_RPN\_ROIS                  & True                                                                                                                                                                                   \\ \hline
		VALIDATION\_STEPS               & 30                                                                                                                                                                                     \\ \hline
		WEIGHT\_DECAY                   & 0.0001  
		\\ \hline 
	\caption{Konfigurasi Model menggunakan Resnet-50 }
	\label{tab:conf-resnet50}
\end{longtable}

Setelah dilakukan serangkaian proses training yang memakan waktu sekitar 3 jam 40 menit 24 detik didapatkan \textit{output} berupa \textit{model file}  dengan format \textit{h5} yang mempunyai ukuran 170.9 MB. \textit{Training loss} terendah yang berhasil dicapai dengan menggunakan \textit{backbone} Resnet-50 (pada \textit{epoch} ke 242) adalah 0.4061 dengan rincian \textit{training bounding box loss} sebesar 0.04083, \textit{training classification loss} sebesar 0.02268 serta \textit{training mask loss} sebesar 0.139 (dimana $L=L_{bbox}+L_{cls}+L_{mask}$). Gambar \ref{fig:resnet50-training} merupakan grafik yang menunjukkan perubahan \textit{training loss, training bounding box loss, training classification loss,} serta \textit{training mask loss} dari \textit{epoch} 1 sampai 300.

\begin{figure}[ht]
	\centering
	\includegraphics[scale=0.4]{gambar/resnet50-training.png}
	\caption{Grafik Perubahan \textit{Training Loss}}
	\label{fig:resnet50-training}
\end{figure}

Sedangkan pada saat proses \textit{validation} sendiri \textit{Loss} terendah yang berhasil dicapai pada \textit{epoch} ke 266 dengan nilai sebesar 0.3653 dengan rincian \textit{validation bounding box loss} sebesar 0.4556, \textit{validation classification loss} sebesar 0.01912 serta \textit{validation mask loss} sebesar 0.1519 (dimana $L=L_{bbox}+L_{cls}+L_{mask}$). Namun untuk \textit{validation bounding box loss} terendah berada pada \textit{epoch} ke 260 dengan nilai sebesar 0.4203 sedangkan \textit{validation classification loss} terendah pada \textit{epoch} ke 210 dengan nilai 0.01525 serta \textit{validation mask loss} terendah pada \textit{epoch} ke 201 dengan nilai 0.1439. Gambar \ref{fig:resnet50-val} merupakan grafik yang menunjukkan perubahan \textit{validation loss, validation bounding box loss, validation classification loss,} serta \textit{validation mask loss} dari \textit{epoch} 1 sampai 300.

\begin{figure}[ht]
	\centering
	\includegraphics[scale=0.4]{gambar/resnet50-val.png}
	\caption{Grafik Perubahan \textit{Validation Loss}}
	\label{fig:resnet50-val}
\end{figure}

Selain menggunakan \textit{Loss Function} untuk mengukur peforma hasil \textit{training} yang sudah dilakukan, digunakan juga \textit{mean Average Precision (mAP)}. \textit{Precision} sendiri merupakan fungsi untuk menggambarkan tingkat keakuratan antara data yang diminta dengan hasil prediksi yang diberikan oleh model. Maka, \textit{precision} merupakan rasio prediksi benar positif (TP) dibandingkan dengan keseluruhan hasil yang diprediksi positif (TP dan FP). Rumus untuk mencari \textit{Precision} adalah sebagai berikut :
\begin{equation}
	Precision = \frac{TP}{TP+FP} 
\end{equation}

Perhitungan \textit{mAP} pada penelitian ini dilakukan setiap 5 \textit{epoch} sekali, karena jika dilakukan setiap \textit{epoch} akan memerlukan \textit{training time} yang lebih lama serta \textit{resource hardware} yang diperlukan lebih besar. Nilai \textit{mAP} tertinggi didapatkan pada \textit{epoch} ke 220 sebesar 94.92. Gambar \ref{fig:resnet50-map} merupakan grafik yang menunjukan perubahan \textit{validation mean Average Precision} dari \textit{epoch} 1 sampai 300. 

\begin{figure}[h!]
	\centering
	\includegraphics[scale=0.7]{gambar/resnet50-map.png}
	\caption{Grafik Perubahan \textit{Validation mAP}}
	\label{fig:resnet50-map}
\end{figure}

\subsection{Resnet 101}
\label{resnet101}

Resnet-101 merupakan salah satu \textit{Image Classification} dari CNN dengan jumlah layer sebanyak 101. Gambar \ref{fig:resnet101-arch} adalah arsitektur dari Resnet-101.

\subsection{MobileNet V1}
\label{mobilenetv1}

\subsection{MobileNet V2}
\label{mobilenetv2}

\section{Pengujian Jenis Teknik Validasi}
\label{sec:pengujian-validation}

\subsection{Pengujian dengan Pemisahan Data Validasi}
\label{normal-validasi}

\subsection{Pengujian dengan \textit{K Fold Cross Validation}}
\label{cross-validasi}