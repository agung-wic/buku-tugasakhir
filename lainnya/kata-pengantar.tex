\begin{center}
  \Large
  \textbf{KATA PENGANTAR}
\end{center}

% Ubah paragraf-paragraf berikut dengan isi dari kata pengantar

Puji dan syukur kehadirat Tuhan Yang Maha Esa atas segala karunia-Nya, penulis  dapat menyelesaikan penelitian ini dengan judul \textbf{Deteksi Pejalan Kaki pada \textit{Zebra Cross} untuk Peringatan Dini Pengendara Mobil menggunakan \textit{Mask R-CNN}.}

Penelitian ini disusun dalam rangka pemenuhan bidang riset di Departemen Teknik Komputer ITS, sera digunakan sebagai persyaratan menyelesaikan pendidikan Sarjana. Penelitian ini dapat diselesaikan tidak lepas dari bantuan berbagai pihak. Oleh karena itu, penulis mengucapkan terimakasih kepada:

\begin{enumerate}[nolistsep]
  \item Keluarga, Ibu, Bapak dan Saudara tercinta yang telah memberikan dorongan baik secara spiritual dan material dalam penyelesaian buku penelitian ini.
  \item Bapak Dr. Supeno Mardi Susiki Nugroho, ST., MT. selaku Kepala Departemen Teknik Komputer, Fakultas Teknologi Elektro dan Informatika Cerdas, Institut Teknologi Sepuluh Nopember. 
  \item Bapak Prof. Dr. Ir. Mauridhi Hery Purnomo, M.Eng. selaku dosen pembimbing I dan Bapak Dr. Eko Mulyanto Yuniarno, S.T., M.T. selaku dosen pembimbing II yang selalu memberikan arahan selama mengerjakan penelitian tugas akhir ini.
  \item Bapak-ibu dosen pengajar Departemen Teknik Komputer, atas pengajaran dam bimbingan yang diberikan kepada penulis.
  \item Seluruh teman-teman dari angkatan e57, Teknik Komputer, Laboratorium B401 dan B201 Teknik Komputer ITS serta Saturasi ITS.
\end{enumerate}

Kesempurnaan hanya milik Allah SWT, untuk itu penulis memohon segenap kritik dan saran yang membangun. Semoga penelitian ini dapat memberikan manfaat bagi kita semua. Amin.


\begin{flushright}
  \begin{tabular}[b]{c}
    % Ubah kalimat berikut dengan tempat, bulan, dan tahun penulisan
    Surabaya, 9 Juli 2021\\
    \\
    \\
    \\
    \\
    % Ubah kalimat berikut dengan nama mahasiswa
    Agung Wicaksono
  \end{tabular}
\end{flushright}
